\begin{center}
  \large\textbf{ABSTRACT}
\end{center}

\addcontentsline{toc}{chapter}{ABSTRACT}

\vspace{2ex}

\begingroup
% Menghilangkan padding
\setlength{\tabcolsep}{0pt}

\noindent
\begin{tabularx}{\textwidth}{l >{\centering}m{3em} X}
  \emph{Name}     & : & \name{}         \\

  \emph{Title}    & : & \engtatitle{}   \\

  \emph{Advisors} & : & 1. \advisor{}   \\
                  &   & 2. \coadvisor{} \\
\end{tabularx}
\endgroup

% Ubah paragraf berikut dengan abstrak dari tugas akhir dalam Bahasa Inggris
\emph{The objective of this study is to develop an autonomous wheelchair control system capable of real-time trajectory tracking. The system integrates the object detection algorithm YOLOv11 and body motion tracking using MediaPipe Pose, with the optimal viewing angle serving as the basis for visual data collection. The camera is positioned on the user's glasses to ascertain the optimal viewing angle for detecting and tracking user movement. This system is designed to operate wirelessly using the ESP32 module, thereby facilitating flexibility and efficiency in controlling wheelchair movement. Testing was conducted in a controlled environment to ensure the accuracy and speed of user movement detection and tracking. The results demonstrated that the system can follow user movement with precision and responsiveness, thereby making a substantial contribution to the advancement of intelligent and independent health mobility technology.}

% Ubah kata-kata berikut dengan kata kunci dari tugas akhir dalam Bahasa Inggris
\emph{Keywords: Autonomous Wheelchair, YOLOv11, MediaPipe Pose, Optimal Viewing Angle, Motion Detection, Wireless Control, Health Mobility.}
