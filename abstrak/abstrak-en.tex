\begin{center}
  \large\textbf{ABSTRACT}
\end{center}

\addcontentsline{toc}{chapter}{ABSTRACT}

\vspace{2ex}

\begingroup
% Menghilangkan padding
\setlength{\tabcolsep}{0pt}

\noindent
\begin{tabularx}{\textwidth}{l >{\centering}m{3em} X}
  \emph{Name}     & : & \name{}         \\

  \emph{Title}    & : & \engtatitle{}   \\

  \emph{Advisors} & : & \advisor{}   \\
                  % &   & 2. \coadvisor{} \\
\end{tabularx}
\endgroup

% Ubah paragraf berikut dengan abstrak dari tugas akhir dalam Bahasa Inggris
\emph{This study aims to develop an autonomous wheelchair control system that can follow human movement in real-time. This system integrates the object detection algorithm \emph{YOLOv8} and body motion tracking using MediaPipe Pose, utilizing the optimal viewing angle as the basis for visual data collection. The camera is placed on the user's glasses to obtain the optimal viewing angle in detecting and tracking user movement. This system is designed to operate wirelessly using the ESP32 module, providing flexibility and efficiency in controlling wheelchair movement. Testing was conducted in a controlled environment to ensure the accuracy and speed of user movement detection and tracking. The results showed that the system can follow user movement accurately and responsively, making a significant contribution to the development of smarter and more independent health mobility technology.}

% Ubah kata-kata berikut dengan kata kunci dari tugas akhir dalam Bahasa Inggris
\emph{Keywords: Autonomous Wheelchair, YOLOv8, MediaPipe Pose, Optimal Viewing Angle, Motion Detection, Wireless Control, Health Mobility.}
