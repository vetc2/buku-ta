\begin{center}
  \large\textbf{ABSTRAK}
\end{center}

\addcontentsline{toc}{chapter}{ABSTRAK}

\vspace{2ex}

\begingroup
% Menghilangkan padding
\setlength{\tabcolsep}{0pt}

\noindent
\begin{tabularx}{\textwidth}{l >{\centering}m{2em} X}
  Nama Mahasiswa    & : & \name{}         \\

  Judul Tugas Akhir & : & \tatitle{}      \\

  Pembimbing        & : & 1. \advisor{}   \\
                    &   & 2. \coadvisor{} \\
\end{tabularx}
\endgroup

% Ubah paragraf berikut dengan abstrak dari tugas akhir
Penelitian ini bertujuan untuk mengembangkan sistem kendali kursi roda otonom yang dapat mengikuti pergerakan manusia secara real-time. Sistem ini mengintegrasikan algoritma deteksi objek \emph{YOLOv11} dan pelacakan gerakan tubuh menggunakan \emph{MediaPipe Pose}, dengan memanfaatkan sudut pandang optimal sebagai dasar pengambilan data visual. Kamera ditempatkan pada kacamata pengguna untuk mendapatkan sudut pandang yang optimal dalam mendeteksi dan melacak pergerakan pengguna. Sistem ini dirancang agar dapat beroperasi secara nirkabel menggunakan modul \emph{ESP32}, memberikan fleksibilitas dan efisiensi dalam mengendalikan pergerakan kursi roda. Pengujian dilakukan dalam lingkungan terkendali untuk memastikan keakuratan dan kecepatan deteksi serta pelacakan gerakan pengguna. Hasil penelitian menunjukkan bahwa sistem dapat mengikuti pergerakan pengguna dengan akurat dan responsif, memberikan kontribusi signifikan terhadap pengembangan teknologi mobilitas kesehatan yang lebih cerdas dan mandiri.

% Ubah kata-kata berikut dengan kata kunci dari tugas akhir
Kata Kunci: Kursi Roda Otonom, YOLOv11, MediaPipe Pose, Sudut Pandang Optimal, Deteksi Gerakan, Kendali Nirkabel, Mobilitas Kesehatan
