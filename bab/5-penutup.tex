\chapter{PENUTUP}
\label{chap:penutup}

% Ubah bagian-bagian berikut dengan isi dari penutup

\section{Kesimpulan}
\label{sec:kesimpulan}

Berdasarkan hasil pengujian yang telah dilakukan, dapat diambil kesimpulan sebagai berikut:

\begin{enumerate}[nolistsep]
  \item Model YOLOv11 menunjukkan performa deteksi yang tinggi dengan skor mAP mencapai 81.85\% pada threshold 0.5.
  \item Sistem mampu memproses frame secara real-time dengan rata-rata FPS sekitar 40, menunjukkan kecepatan pemrosesan yang memadai untuk aplikasi lapangan.
  \item Pengujian regulator time menunjukkan bahwa 66.17\% respons berada dalam waktu yang relatif singkat, namun 34\% lainnya melebihi 0.5 detik.
  \item Sistem berhasil mengunci target (Locked Detection) sebanyak 4 kali dari 15 percobaan, menghasilkan tingkat keberhasilan sebesar 26.67\%.
  \item Sistem menunjukkan kinerja terbaik pada kondisi pencahayaan optimal, namun kinerja menurun pada kondisi pencahayaan rendah.
  \item Sistem mampu mendeteksi objek dengan jarak kurang dari 1 meter secara konsisten dan mengirimkan instruksi "Stop" untuk mencegah tabrakan.
  \item Sistem memiliki performa yang sangat baik dalam mendeteksi dan mengikuti objek di berbagai skenario, baik saat dalam frame, bergerak maju, maupun berbelok.
  \item Sistem mampu mengikuti objek dengan baik di seluruh kondisi jalur yang diuji, dengan tingkat keberhasilan mencapai 100\%.

\end{enumerate}

\section{Saran}
\label{chap:saran}

Untuk pengembangan lebih lanjut pada penelitian selanjutnya, adapun saran yang bisa diberikan antara lain:

\begin{enumerate}[nolistsep]
  \item Meningkatkan fungsi \texttt{track()} untuk menjaga konsistensi identitas ID objek yang terdeteksi, guna meningkatkan akurasi pelacakan.
  \item Menggunakan pencahayaan tambahan yang tidak menyebabkan overexposure dapat dipertimbangkan untuk meningkatkan kinerja deteksi pada kondisi pencahayaan rendah.
  \item Memberikan optimisasi lebih lanjut pada sistem untuk mengurangi waktu respons yang melebihi 0.5 detik, guna meningkatkan stabilitas dan konsistensi kinerja.
  \item Menuliskan algoritma yang lebih adaptif untuk menangani kondisi luar frame, sehingga dapat mencegah pengiriman instruksi yang tidak relevan atau tidak aman.
  \item Implementasi sistem pada skenario nyata dengan berbagai kondisi operasional untuk menguji dan memastikan keandalan serta keamanan sistem secara menyeluruh.
\end{enumerate}
