\chapter{PENDAHULUAN}
\label{chap:pendahuluan}

% Ubah bagian-bagian berikut dengan isi dari pendahuluan

\section{Latar Belakang}
\label{sec:latarbelakang}

Teknologi \emph{Internet of Things (IoT)} dan \emph{deep learning} semakin banyak diadopsi dalam pengembangan sistem kendali otonom, khususnya untuk aplikasi kesehatan dan mobilitas. Kursi roda otonom merupakan salah satu solusi yang dapat meningkatkan kualitas hidup penyandang disabilitas dengan memberikan kemandirian dalam bergerak. Salah satu teknologi utama yang dapat mendukung pengembangan kursi roda otonom adalah modul \emph{ESP32}. Berdasarkan penelitian yang dilakukan oleh \emph{Ekatama} (2024), \emph{ESP32} mampu mengendalikan perangkat secara nirkabel dengan memanfaatkan konektivitas \emph{Wi-Fi} dan \emph{Bluetooth}, memberikan fleksibilitas yang lebih besar dalam mengontrol kursi roda. Penggunaan modul ini memungkinkan pengguna untuk mengontrol kursi roda secara efektif tanpa perlu mengandalkan kontrol manual yang terbatas \cite{ekatama2024perancangan}.

Selain aspek kendali nirkabel, sistem otonom yang mengikuti pergerakan pengguna membutuhkan teknologi yang mampu mendeteksi gerakan tubuh secara real-time. Penelitian oleh \emph{Wijaya et al.} (2022) telah menunjukkan keandalan algoritma \emph{YOLO V3} dalam mendeteksi objek secara cepat dan akurat, yang dapat digunakan untuk melacak pergerakan manusia. Namun, dalam penelitian ini, algoritma \emph{YOLOv8}, versi terbaru dari \emph{YOLO}, diintegrasikan dengan \emph{MediaPipe Pose}, sebuah framework yang dapat mendeteksi dan melacak posisi tubuh manusia. Kombinasi kedua teknologi ini memungkinkan sistem untuk secara akurat mengikuti gerakan pengguna berdasarkan sudut pandang optimal yang diperoleh dari kamera yang dipasang pada kacamata pengguna. Dengan pendekatan ini, kursi roda dapat mengikuti pergerakan pengguna secara alami, mendukung otonomi yang lebih tinggi \cite{wijaya2022deteksi}.

Sistem kursi roda otonom yang dikembangkan dalam penelitian ini memanfaatkan modul \emph{ESP32} sebagai perangkat keras utama yang mengontrol keseluruhan sistem secara nirkabel. Dalam penelitian yang dilakukan oleh \emph{Narwaria et al.} (2024), \emph{ESP32-CAM} telah terbukti mampu menangkap dan mengolah data visual dengan efisiensi tinggi, yang dapat diterapkan untuk berbagai aplikasi berbasis \emph{IoT}. Sistem yang dikembangkan tidak hanya dirancang untuk mendeteksi objek, tetapi lebih berfokus pada pelacakan gerakan tubuh pengguna melalui teknologi \emph{MediaPipe Pose}. Dengan integrasi ini, sistem dapat memastikan bahwa pergerakan kursi roda tetap sejalan dengan pergerakan pengguna tanpa memerlukan input manual tambahan \cite{10696374}.

\section{Permasalahan}
\label{sec:permasalahan}

Dari permasalahan tersebut maka pemantauan manusia pada kursi roda dengan fokus pada kemandirian menghadapi beberapa tantangan utama:

\begin{enumerate}[nolistsep]
      \item \textbf{Keterbatasan Sistem Kursi Roda Otonom Eksisting:} Sistem kursi roda otonom yang ada saat ini mungkin belum mampu mendeteksi dan mengikuti pengguna dengan efisien karena penempatan kamera yang kurang optimal dan penggunaan algoritma deteksi yang kurang canggih.
      \item \textbf{Tantangan dalam Deteksi dan Pelacakan Gerakan Manusia secara Real-Time:} Kesulitan dalam mendeteksi dan melacak pergerakan manusia secara akurat dan real-time, terutama dalam lingkungan yang dinamis, akibat keterbatasan algoritma dan perangkat keras.
      \item \textbf{Integrasi Algoritma Deteksi Lanjutan dengan Sudut Pandang Optimal:} Tantangan dalam mengintegrasikan algoritma deteksi seperti YOLOv8 dan MediaPipe Pose dengan sistem pengambilan gambar yang memiliki sudut pandang optimal untuk meningkatkan akurasi dan responsivitas kursi roda.
\end{enumerate}

\section{Tujuan}
\label{sec:Tujuan}

Tujuan dari penelitian ini adalah mengembangkan sistem kursi roda otonom yang dapat mengikuti pergerakan pengguna secara real-time, dengan rancangan dan implementasi sistem kendali kursi roda yang dapat mengikuti pergerakan pengguna dengan memanfaatkan sudut pandang optimal dari kamera yang ditempatkan pada kacamata.

\section{Batasan Masalah}
\label{sec:batasanmasalah}

Batasan-batasan dari penelitian ini diharapkan memberikan kontribusi signifikan dalam bidang pemantauan manusia pada kursi roda dengan fokus pada kemandirian dan kontrol otomatis, dengan:

\begin{enumerate}[nolistsep]
      \item \textbf{Penempatan Kamera Terbatas pada Sudut Pandang Optimal:} Sistem hanya menggunakan kamera yang ditempatkan pada kacamata pengguna, sehingga deteksi terbatas pada objek dan pergerakan yang berada dalam garis pandang pengguna.
      \item \textbf{Penggunaan Algoritma dan Perangkat Keras Tertentu:} Sistem deteksi dan pelacakan dibatasi pada penggunaan algoritma YOLOv8 dan MediaPipe Pose, serta perangkat keras yang mendukung implementasi algoritma tersebut.
      \item \textbf{Lingkungan Operasional Terkendali:} Pengujian dan implementasi sistem dilakukan dalam lingkungan yang terkendali, seperti di dalam ruangan dengan kondisi pencahayaan dan hambatan yang minimal.
      \item \textbf{Keterbatasan Pemrosesan Waktu Nyata:} Kemampuan pemrosesan real-time sistem dibatasi oleh kapasitas perangkat keras yang digunakan, sehingga mungkin tidak optimal dalam situasi dengan kompleksitas tinggi.
\end{enumerate}

\section{Manfaat}
\label{sec:manfaat}

Manfaat pada penelitian ini untuk membuat sistem yang dapat mendeteksi manusia untuk mengontrol gerak dari kursi roda.
